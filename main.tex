\documentclass{article}
\usepackage{amsmath,amssymb,amsthm,stmaryrd,mathtools}
\usepackage{hyperref} 
\usepackage{tikz-cd}
\usepackage{mathpartir} % for nicely typeset inference rules

\newcommand{\Type}{\ensuremath{\mathsf{Type}}}
\newcommand{\defeq}{\vcentcolon\equiv}
\newcommand{\inl}{\mathsf{inl}}
\newcommand{\inr}{\mathsf{inr}}
\newcommand{\caseof}{\mathsf{case}}
\newcommand{\Nat}{\mathsf{Nat}}
\newcommand{\Bool}{\mathsf{Bool}}
\newcommand{\String}{\mathsf{String}}

% handy macros (non-conflicting with yours)
\newcommand{\emptyctx}{\cdot}              % empty context
\newcommand{\ctx}{\mathsf{ctx}}            % "is a well-formed context" judgment
\newcommand{\judg}[3]{#1 \vdash #2 : #3}   % typing judgment \judg{\Gamma}{t}{A}
\newcommand{\jdeq}{\equiv}                 % definitional equality (judgmental)
\newcommand{\teq}[4]{#1 \vdash #2 \jdeq #3 : #4} % \teq{\Gamma}{t}{u}{A}
\newcommand{\dom}{\mathsf{dom}}            % domain of a context

% optional: small, named rule labels
\newcommand{\rulename}[1]{\textsc{#1}}

% ---------- TITLE INFO ----------
\title{Dependent Type Theory for Absolute Beginners \\\vspace{0.2cm} 
\small Created with the help of GPT-5 Pro}
\author{Kimon Fountoulakis}
\date{\today}
% --------------------------------

\begin{document}

\maketitle
\tableofcontents
\newpage

\section{Types and universes}

\subsection*{Types}

In dependent type theory, all objects we manipulate are \emph{terms}, and each term has a \emph{type}.  
For example, a natural number \(0\) has type \(\Nat\), and a boolean value \(\mathsf{true}\) has type \(\Bool\). We use the colon notation \(a : A\) to mean that \(a\) is a term of the type \(A\).  
For example:
\[
0 : \Nat, \quad \mathsf{false} : \Bool, \quad ``hello \ world" : \String.
\]

\subsection*{Universes}

But what about \(\Nat\) and \(\Bool\) themselves?  
They too are objects in the theory and therefore must have a type.  
The type to which all ordinary types belong is called the \emph{universe of types} and is written \(\Type\).  
Thus we write
\[
\Nat : \Type, \qquad \Bool : \Type, \qquad \String : \Type.
\]
Intuitively, \(\Type\) is “the type of all types.”  
However, if we were to write \(\Type : \Type\), we would obtain an inconsistency (Girard’s paradox).  
To avoid this, type theory introduces an infinite hierarchy of universes:
\[
\Type_0,\;\Type_1,\;\Type_2,\;\dots
\]
Each universe is itself a type in the next one, forming a cumulative hierarchy:
\[
\Type_m : \Type_n \quad \text{for } m < n.
\]
Intuitively, $\Type_0$ is the universe of ``small'' types (such as $\Nat$, $\Bool$, etc.), $\Type_1$ is the universe of ``types of small types,'' and so on. ``In practice,'' we often omit universe indices and simply write $\Type$
when the specific level is irrelevant.


\section{Introduction to the notation \texorpdfstring{$\lambda x\!:\!A.\,t$}{λx:A.t}}

The expression
\[
\lambda x\!:\!A.\,t
\]
is called a \emph{lambda abstraction} or \emph{function definition}.  
It denotes a function that takes an input \(x\) of type \(A\) and returns the term \(t\) as its output.

\subsection*{Structure}

The components of the expression are as follows:

\[
\begin{array}{ll}
\lambda & \text{(the ``lambda'' symbol introducing a function);} \\
x & \text{(the variable name of the function argument);} \\
:A & \text{(the type annotation for the variable);} \\
. & \text{(a separator between the argument and the function body);} \\
t & \text{(the function body, possibly depending on \(x\)).}
\end{array}
\]

Hence, the expression
\[
\lambda x\!:\!A.\,t
\]
should be read as
\begin{quote}
``the function that takes an argument \(x\) of type \(A\) and returns \(t\).''  
\end{quote}

\subsection*{Examples}

\paragraph{Identity function.}
\[
\lambda x\!:\!\Nat.\,x
\]
This denotes the function that takes a natural number \(x\) and returns \(x\) itself.  
Its type is \(\Nat \to \Nat\).

\paragraph{Constant function.}
\[
\lambda x\!:\!\Nat.\,0
\]
This function ignores its input and always returns \(0\).  
Type: \(\Nat \to \Nat\).

\paragraph{Boolean negation.}
\[
\lambda b\!:\!\Bool.\;
\caseof(b,\;
  \lambda\_.\,\mathsf{false},\;
  \lambda\_.\,\mathsf{true})
\]
This represents the Boolean negation function, of type \(\Bool \to \Bool\). The expression
\[
\caseof(b,\;
  \lambda\_.\,\mathsf{false},\;
  \lambda\_.\,\mathsf{true})
\]
is the \emph{eliminator} (or destructor) for the boolean type.
It performs a case distinction depending on whether \(b\) is
\(\mathsf{true}\) or \(\mathsf{false}\):
\[
\caseof(b,\;
  \lambda\_.\,t_1,\;
  \lambda\_.\,t_2)
\;\text{means}\;
\begin{cases}
t_1 & \text{if } b = \mathsf{true},\\
t_2 & \text{if } b = \mathsf{false}.
\end{cases}
\]
The underscore ``\_'' denotes an \emph{ignored argument};
for instance, \(\lambda\_.\,\mathsf{false}\) is the function that always returns
\(\mathsf{false}\) regardless of its input.
Hence
\[
\lambda b\!:\!\Bool.\;
\caseof(b,\;
  \lambda\_.\,\mathsf{false},\;
  \lambda\_.\,\mathsf{true})
\]
defines the boolean negation function, i.e.\ a function that returns
\(\mathsf{false}\) when \(b\) is true and \(\mathsf{true}\) when \(b\) is false.

\subsection*{Function application and computation}

If \(f\) is a lambda abstraction and \(a\) is an argument of the appropriate type,  
we apply \(f \jdeq \lambda x\!:\!A.\,t \) to \(a\) by writing
\[
f\,a.
\]
The result is obtained by substituting \(a\) for all free occurrences of \(x\) in the body \(t\):
\[
(\lambda x\!:\!A.\,t)\,a \;\jdeq\; t[a/x].
\]
This is called the \emph{beta reduction} (\(\beta\)-reduction), and it formalizes the usual idea of function application.

\paragraph{Example.}
\[
(\lambda x\!:\!\Nat.\,x + 1)\,3 \;\jdeq\; 3 + 1 \;\jdeq\; 4.
\]
Intuitively, the lambda abstraction defines a function, and applying it ``plugs in'' the argument.

\subsection*{Relation to ordinary mathematics}

The notation
\[
\lambda x\!:\!A.\,t
\]
corresponds to the mathematical function expression
\[
x \mapsto t(x).
\]
The lambda notation, however, treats functions as \emph{first-class terms}—they can be
passed as arguments, returned as results, and manipulated like any other expression.


% \[
% \Type_0 : \Type_1 : \Type_2 : \cdots
% \]
% Each universe \(\Type_i\) is an element of the next one, but not of itself.  
% In most practical situations, we omit universe index that a term belongs to and simply write \(\Type\). 

% \emph{Examples:}
% \[
% \begin{aligned}
% \Nat &:\Type_0, &\text{the type of natural numbers;}\\
% \Bool &:\Type_0, &\text{the type of booleans;}\\
% \String &:\Type_0, &\text{the type of strings;}\\
% % \mathbf{1},\,\mathbf{0} &:\Type_0, &\text{the unit and empty types.}
% \end{aligned}
% \]
% Generally, writing \(A : \Type\) means that \(A\) is itself a type.

\section{Contexts}

\subsection*{The empty context}
The symbol \(\emptyctx\), often written simply as \(\cdot\), denotes the \emph{empty context}.
It represents the starting point of all contexts---a situation with no assumptions.
% Formally, it satisfies the judgment
% \[
% \cdot \vdash \ctx,
% \]
% which reads “the empty context is well-formed.”

\subsection*{General contexts}

First, some terminology, \(x : A\) is a called a \emph{variable declaration}.
A context is simply a finite list of typed variable declarations.  
For example:
\[
\emptyctx, \qquad
x : \Nat, \qquad
x : \Nat,\, y : \Bool, \qquad
x : \Nat,\, y : \Bool,\, z : \String.
\]
These are four different contexts. The first, i.e., $\cdot$, denotes the empty context. The second includes only one declaration, the third includes two declarations and the forth includes three declarations. Each larger context is obtained from a smaller one by adding one more declaration.

\subsection*{Extension of a context}
Let $\Gamma$ be a context, then \(\Gamma, x : A\) is the \emph{extension} of \(\Gamma\) with a new declaration.

\paragraph{Example of construction.}
\[
\Gamma_0 = \emptyctx, \qquad
\Gamma_1 = \Gamma_0, x : \Nat, \qquad
\Gamma_2 = \Gamma_1, y : \Bool, \qquad
\Gamma_3 = \Gamma_2, z : \String.
\]
Hence
\[
\Gamma_3 = x : \Nat,\, y : \Bool,\, z : \String.
\]
This shows how a recursive definition generates all finite contexts.

\subsection*{Inductive definition of contexts}

Formally, we define the \emph{syntax of contexts} using the notation
\[
\Gamma ::= \emptyctx \;\mid\; \Gamma,\, x : A.
\]
The symbol \texttt{::=} is read as “is defined as,” and the vertical bar “\texttt{|}” means “or.”  
Thus, this definition should be read as:

\begin{quote}
A context \(\Gamma\) is either the empty context \(\emptyctx\),  
or an existing context extended by a new variable declaration \(x : A\).
\end{quote}

This is a \emph{recursive} or \emph{inductive} definition:  
\begin{itemize}
  \item The \emph{base case} says that the empty context \(\emptyctx\) is a valid context.
  \item The \emph{inductive case} says that if \(\Gamma\) is a valid context and \(A\) is a type, then \(\Gamma, x : A\) is also a valid context.
\end{itemize}

The domain of a context $\Gamma$ is denoted by \(\dom(\Gamma)\), and it is the set of variables declared in \(\Gamma\).

\section{Judgments}

A \emph{judgment} is a basic assertion of the form we can derive in the theory.  

We will use the following judgment forms throughout:
\begin{quote}
  \(\judg{\Gamma}{t}{A}\), reads “under context \(\Gamma\), the term \(t\) has type \(A\).
\end{quote}
and
\begin{quote}
  \(\judg{\Gamma}{a \equiv b }{A}\), reads “under context \(\Gamma\), the terms \(a\) and \(b\) are judgementally equal at type \(A\),
\end{quote}
and 
\begin{quote}
  \(\Gamma \vdash \ctx\), reads “context \(\Gamma\) is well-formed.
\end{quote}
Judgements are derived using inference rules, which we will discuss later. For now, let's explain judgementally equivalent means and what a well-formed context means. 
% Here, \(\vdash\) is the \emph{turnstile} symbol separating assumptions (on the left) from a conclusion (on the right).

% \paragraph{Example (typing).}
% With \(\Nat : \Type\) and \(\mathbf{1}:\Type\) (unit), the following are typing judgments:
% \[
% \emptyctx \vdash \star : \mathbf{1}
% \qquad\text{and}\qquad
% x:\Nat \vdash x : \Nat.
% \]
% The left judgement means that from the empty context we conclude that $\star$ is a term of the unit type. The right judgement is trivial. It means that if we assume that $x$ in a natural number, then we can conclude that $x$ is a natural number.


% \subsection*{What does it mean for a judgment to be derivable?}

% When we say that a judgment is \emph{derivable from the rules} (or that \emph{there exists a proof tree}), we mean that the judgment can be obtained step by step by applying the inference rules of the system.  
% Each inference rule allows us to infer a new judgment from some existing ones.  
% A \emph{derivation} is simply a finite tree of such applications, ending with the judgment we want to justify.

% \paragraph{Intuitive picture.}
% The inference rules are the building blocks of reasoning in the type system.
% Each rule says something of the form:
% \[
% \text{if these premises hold, then you may conclude this judgment.}
% \]
% A judgment is said to be \emph{derivable} if there exists a finite sequence (tree)
% of applications of these rules that ends with that judgment at its root.
% If no such tree can be built, the judgment is \emph{not derivable} (and therefore incorrect).

% \paragraph{Example: the identity function.}

% Let us show that the following judgment is derivable:
% \[
% \judg{\emptyctx}{\lambda x\!:\!\Nat.\,x}{\Nat \to \Nat}.
% \]

% \noindent
% We will build the derivation (proof tree) step by step.

% \begin{enumerate}
%   \item Our goal is \(\judg{\emptyctx}{\lambda x:\Nat.\,x}{\Nat \to \Nat}\).
%   The only rule that concludes something of this form is the \rulename{\(\to\)-Intro} rule:
%   \[
%   \inferrule*[right=\rulename{\(\to\)-Intro}]{
%     \judg{\Gamma,\,x\!:\!A}{t}{B}
%   }{
%     \judg{\Gamma}{\lambda x\!:\!A.\,t}{A \to B}
%   }
%   \]
%   Thus, to apply this rule, we must show
%   \[
%   \judg{x:\Nat}{x}{\Nat}.
%   \]

%   \item In the context \(x:\Nat\), we can use the \rulename{Var} rule:
%   \[
%   \inferrule*[right=\rulename{Var}]{
%     (x:\Nat)\in(x:\Nat)
%   }{
%     \judg{x:\Nat}{x}{\Nat}
%   }
%   \]
%   This closes the branch of the derivation tree, since the premise is trivially satisfied.

%   \item Using this result as the premise of \rulename{\(\to\)-Intro}, we conclude:
%   \[
%   \judg{\emptyctx}{\lambda x:\Nat.\,x}{\Nat \to \Nat}.
%   \]
% \end{enumerate}

% \paragraph{The complete derivation tree.}
% \[
% \inferrule*[right=\rulename{\(\to\)-Intro}]{
%   \inferrule*[right=\rulename{Var}]{
%     (x:\Nat)\in(x:\Nat)
%   }{
%     \judg{x:\Nat}{x}{\Nat}
%   }
% }{
%   \judg{\emptyctx}{\lambda x\!:\!\Nat.\,x}{\Nat \to \Nat}
% }
% \]

% \paragraph{Why this matters.}
% This tree is the \emph{derivation} proving that the judgment
% \(\judg{\emptyctx}{\lambda x\!:\!\Nat.\,x}{\Nat \to \Nat}\) is correct according to the rules.
% The existence of such a tree certifies that the judgment is derivable.

% \paragraph{Contrast: a non-derivable case.}
% Consider the judgment
% \[
% \judg{\emptyctx}{\lambda x\!:\!\Bool.\,x}{\Nat \to \Nat}.
% \]
% Attempting to apply the \rulename{\(\to\)-Intro} rule requires
% \(\judg{x:\Bool}{x}{\Nat}\),
% but the variable rule only gives
% \(\judg{x:\Bool}{x}{\Bool}\).
% There is no way to complete the derivation.
% Hence, this judgment is \emph{not derivable}—it is \emph{incorrect}.

\subsection*{Definition (judgmental/definitional equality).}
The judgment \(\teq{\Gamma}{a}{b}{A}\) means that
\(a\) and \(b\) are \emph{judgmentally equal at type \(A\)}. The equallity holds by computation and definitional unfolding.

\paragraph{Example.}
Let
\[
f \;\defeq\; \lambda x\!:\!\Nat.\, x + 1.
\]
Then by definition,
\[
f\,2 \;\jdeq\; (\lambda x\!:\!\Nat.\,x + 1)\,2 \;\jdeq\; 2 + 1 \;\jdeq\; 3.
\]
Hence we can write the judgment
\[
\judg{\emptyctx}{f\,2 \jdeq 3}{\Nat},
\]
which reads: “in the empty context, \(f\,2\) and \(3\) are judgmentally equal at type \(\Nat\).”

\subsection*{Well-formed contexts \texorpdfstring{(\(\Gamma \vdash \ctx\))}{(Gamma |- ctx)}}

We write the \emph{well-formedness} judgment
\[
\Gamma \vdash \ctx
\]
and read it as “\(\Gamma\) is a well-formed context.” The symbol \(\ctx\) is a
fixed tag (a nullary predicate) used only on the right of \(\vdash\) to denote
this judgment; it is not a type.

\paragraph{What does it mean for a context to be well-formed?}

Intuitively, a context \(\Gamma\) is \emph{well-formed} if every declaration inside it makes sense:  
each type appearing in a declaration is itself already a valid type in the smaller context preceding it. Formally, recall that a context is a sequence of variable declarations:
\[
\Gamma \defeq x_1 : A_1,\, x_2 : A_2,\, \dots,\, x_n : A_n.
\]
We say that such a context is \emph{well-formed}, written
\[
\Gamma \vdash \ctx,
\]
if and only if the following recursive conditions hold:
\begin{itemize}
  \item The empty context is well-formed:
    \[
    \emptyctx \vdash \ctx.
    \]
  \item If \(\Gamma \vdash \ctx\) and the type \(A\) is well-formed under \(\Gamma\),
    i.e.\ \(\judg{\Gamma}{A}{\Type}\), and the variable \(x\) is fresh
    (\(x \notin \dom(\Gamma)\)),  
    then the extended context \(\Gamma, x:A\) is well-formed:
    \[
    \Gamma, x:A \vdash \ctx.
    \]
\end{itemize}

Intuitively, this means that the declarations in a context must be arranged so
that each type depends only on variables that have been declared earlier.

\paragraph{Example.}
\[
x:\Nat,\, y:\Bool,\, z:\String \vdash \ctx
\]
is well-formed, since each type (\(\Nat\), \(\Bool\), \(\String\)) is already a valid
type in the previous context.

\paragraph{Dependent example.}
\[
x:\Nat,\, y: (\Bool, x) \vdash \ctx
\]
is also well-formed, because
\((\Bool, x)\) is a type depending on \(x\), and \(x\) has already been declared.

\paragraph{Non-example.}
\[
y:(\Bool, x),\, x:\Nat \not\vdash \ctx,
\]
because \(y\)’s type refers to \(x\), but \(x\) has not yet been declared at that point.

\paragraph{In summary.}
A context is well-formed precisely when every variable declaration in it is
meaningful in the smaller context built from the declarations before it.  
This ensures that type dependencies are acyclic and well-scoped.

\section{Reading inference rules and the rule bar}

An \emph{inference rule} has the schematic form
\[
\inferrule*[right=\rulename{Name}]{
\textit{premise}_1 \quad \cdots \quad \textit{premise}_n
}{
\textit{conclusion}
}
\]
The long \emph{horizontal rule bar} separates premises (above) from the conclusion (below).  
If there are no premises, the rule is an \emph{axiom} (always available). The inference rule is used to derive judgements. 

% It is important to note that the conclusion is not an actual proof. It is a \emph{definitional part of the typing relation}. In the same way, in Haskell, OCaml or TypeScript, the compiler does not \emph{prove} that \texttt{Int -> Bool} is a type; rather, the language definition \emph{declares} that whenever \texttt{Int} and \texttt{Bool} are types, the function type between them is also a type.


% An inference rule with a horizontal bar is a \emph{meta-level typing rule}---it tells the ``compiler for the calculus'' which judgements it may conclude from which premises. For example, the arrow \emph{formation} rule
% \[
% \inferrule*[right=\rulename{Form}]{
%   \judg{\Gamma}{A}{\Type} \quad \judg{\Gamma}{B}{\Type}
% }{
%   \judg{\Gamma}{A \to B}{\Type}
% }
% \]
% plays exactly the role that a language specification plays for a real compiler: \emph{if} $A$ and $B$ are known types in $\Gamma$, \emph{then} the type checker is allowed to accept $A \to B$ as a type in the same context.

% \paragraph{Another example.}
% \begin{itemize}
%   \item \emph{Unit introduction.}
%   \[
%   \inferrule*[right=\rulename{1-Intro}]{
%   }{
%   \judg{\Gamma}{\star}{\mathbf{1}}
%   }
%   \]
% \end{itemize}
% The Unit introduction example reads as ``without any assumptions and under any context $\Gamma$ we conclude that $\star$ is of the unit type''.

\subsection*{Derivation of the empty context}

\medskip
\noindent
\emph{Empty context.}
\[
\inferrule*[right=\rulename{Ctx-Empty}]{
}{
\emptyctx \vdash \ctx
}
\]

\subsection*{Derivation of an extension}
First, we provide one useful definition.

\paragraph{Domain and freshness.}
The set of variables declared in \(\Gamma\) is its \emph{domain}, written
\(\dom(\Gamma)\). It is defined inductively by
\[
\dom(\emptyctx) \defeq \emptyset,
\qquad
\dom(\Gamma, x:A) \defeq \dom(\Gamma) \cup \{x\}.
\]
We write \(x \notin \dom(\Gamma)\) to express that \(x\) is \emph{fresh} for \(\Gamma\).
We use the usual membership notation \((x:A)\in\Gamma\) to mean that the
declaration \(x:A\) occurs somewhere in \(\Gamma\).

\medskip
\noindent
\emph{Extension.}
\[
\inferrule*[right=\rulename{Ctx-Ext}]{
\Gamma \vdash \ctx
\quad
\judg{\Gamma}{A}{\Type}
\quad
x \notin \dom(\Gamma)
}{
\Gamma, x:A \vdash \ctx
}
\]
The side condition \(x \notin \dom(\Gamma)\) enforces that variable names in a context are pairwise distinct. Sometimes, the side condition is not mentioned explicitly as a premise.

\subsection*{Derivation of variables}
From \(\Gamma \vdash \ctx\) and \((x:A)\in\Gamma\) we may derive the trivial
\emph{variable rule}:
\[
\inferrule*[right=\rulename{Var}]{
\Gamma \vdash \ctx
\quad
(x:A)\in\Gamma
}{
\judg{\Gamma}{x}{A}
}
\]

% \subsection*{Examples and a non-example}

% Assume we have the closed type declarations \(\emptyctx \vdash \Nat:\Type\),
% \(\emptyctx \vdash \Bool:\Type\), and \(\emptyctx \vdash \String:\Type\).
% Then the following derivations show well-formed contexts:

% \medskip
% \noindent
% \emph{Example 1.} \(\emptyctx \vdash \ctx\) by \rulename{Ctx-Empty}.

% \medskip
% \noindent
% \emph{Example 2.}
% \[
% \inferrule*[right=\rulename{Ctx-Ext}]{
% \emptyctx \vdash \ctx
% \quad
% \emptyctx \vdash \Nat:\Type
% \quad
% x \notin \dom(\emptyctx)
% }{
% x:\Nat \vdash \ctx
% }
% \]

% \medskip
% \noindent
% \emph{Example 3.}
% \[
% \inferrule*[right=\rulename{Ctx-Ext}]{
% x:\Nat \vdash \ctx
% \quad
% \judg{x:\Nat}{\Bool}{\Type}
% \quad
% y \notin \dom(x:\Nat)
% }{
% x:\Nat,\, y:\Bool \vdash \ctx
% }
% \]

% \medskip
% \noindent
% \emph{Non-example (duplicate name).}
% \[
% x:\Nat,\, x:\Bool \not\vdash \ctx
% \]
% since $x \in \dom(x:\Bool)$ violates $x\notin\dom(x:\Nat)$, meaning that $x$ is not fresh.

\subsection*{Derivation of judgements}

A \emph{derivation} of a judgment is a tree built from inference rules, with the
judgment we want to justify placed at the root. Each node of the tree is an
application of a rule whose premises appear above it and whose conclusion
appears below the horizontal line. For example, using the rules defined so far, we can derive the judgment
\[
\judg{\emptyctx}{x}{\mathbf{1}},
\]
where $\mathbf{1}$ is the \textit{unit type.} The unit type, written \(\mathbf{1} : \Type_0\), is a type with exactly one term, commonly written \(\star\):
\[
\mathbf{1} : \Type_0, \qquad \star : \mathbf{1}.
\]
The full derivation tree is:

\[
\resizebox{\linewidth}{!}{$
  \inferrule*[right=\rulename{Var}]{
    \inferrule*[right=\rulename{Ctx-Ext}]{
      \inferrule*[right=\rulename{Ctx-Empty}]{ }{ \emptyctx \vdash \ctx }
      \quad
      \inferrule*[right=\rulename{\(\mathbf{1}\)-Form}]{ }{ \judg{\emptyctx}{\mathbf{1}}{\Type_0} }
      \quad
      x \notin \dom(\emptyctx)
    }{
      x:\mathbf{1} \vdash \ctx
    }
    \quad
    (x:\mathbf{1}) \in (x:\mathbf{1})
  }{
    \judg{x:\mathbf{1}}{x}{\mathbf{1}}
  }
$}
\]

This tree reads bottom-up as follows:
\begin{itemize}
  \item By \rulename{Ctx-Empty}, the empty context is well-formed.
  \item By \rulename{\(\mathbf{1}\)-Form}, the unit type \(\mathbf{1}\) is a type in the lowest universe \(\Type_0\).
  \item By \rulename{Ctx-Ext}, extending the empty context with \(x:\mathbf{1}\) gives a well-formed context \(x:\mathbf{1}\vdash \ctx\).
  \item By \rulename{Var}, from \(x:\mathbf{1}\vdash \ctx\) we can derive \(\judg{x:\mathbf{1}}{x}{\mathbf{1}}\).
\end{itemize}

Hence the complete derivation establishes
\[
\judg{\emptyctx}{x}{\mathbf{1}}.
\]

\subsection*{Derivation of well-formedness}

We illustrate how to apply the context rules
\[
\inferrule*[right=\rulename{Ctx-Empty}]{ }{\emptyctx \vdash \ctx}
\qquad
\inferrule*[right=\rulename{Ctx-Ext}]{\Gamma \vdash \ctx \quad \judg{\Gamma}{A}{\Type} \quad x \notin \dom(\Gamma)}{\Gamma,x\!:\!A \vdash \ctx}
\]
to derive a well-formed context. 

\paragraph{Comment.} Often, $\Gamma \vdash \ctx$ is not added as a premise, because we already assume $\judg{\Gamma}{A}{\Type}$ in the premise list; and for this premise to be valid, it must already be the case that $\Gamma \vdash \ctx$.


Let's now proceed to a few examples. Throughout, assume the base types are available in the empty context:
\[
\emptyctx \vdash \Nat:\Type,\qquad \emptyctx \vdash \Bool:\Type,\qquad \emptyctx \vdash \String:\Type.
\]

\paragraph{Example 1 (non-dependent).}
Check \(x:\Nat,\, y:\Bool \vdash \ctx\).

\[
\resizebox{\linewidth}{!}{$
\inferrule*[right=\rulename{Ctx-Ext}]{
  \inferrule*[right=\rulename{Ctx-Ext}]{
    \inferrule*[right=\rulename{Ctx-Empty}]{}{ \emptyctx \vdash \ctx }
    \and \judg{\emptyctx}{\Nat}{\Type}
    \and x \notin \dom(\emptyctx)
  }{ x:\Nat \vdash \ctx }
  \and \judg{x:\Nat}{\Bool}{\Type}
  \and y \notin \dom(x:\Nat)
}{
  x:\Nat,\, y:\Bool \vdash \ctx
}
$}
\]
Reading bottom-up:
start with \(\emptyctx\) via \rulename{Ctx-Empty};
extend by \(x:\Nat\) using \(\judg{\emptyctx}{\Nat}{\Type}\);
then extend by \(y:\Bool\) using \(\judg{x:\Nat}{\Bool}{\Type}\) and freshness.

\paragraph{Non-example (dependency out of order).}
Consider \(y:(\Bool,x),\, x:\Nat\).
Attempting \rulename{Ctx-Ext} on the first declaration requires
\(\judg{\emptyctx}{(\Bool,x)}{\Type}\).
But in \(\emptyctx\) there is no variable \(x:\Nat\), so we cannot derive
\(\judg{\emptyctx}{x}{\Nat}\), hence \(\judg{\emptyctx}{(\Bool,x)}{\Type}\) fails.
Therefore
\[
y:(\Bool,x),\, x:\Nat \not\vdash \ctx.
\]

\paragraph{Non-example (duplicate name).}
Consider \(x:\Nat,\, x:\Bool\). The second extension violates freshness since
\(x \in \dom(x:\Nat)\). Thus the side condition \(x \notin \dom(\Gamma)\) fails and
\[
x:\Nat,\, x:\Bool \not\vdash \ctx.
\]

\medskip
\noindent
These derivations show that checking \(\Gamma \vdash \ctx\) reduces \emph{recursively}
to (i) checking the smaller prefix is well-formed, (ii) checking the new
declaration’s type is a type in that prefix, and (iii) enforcing freshness.


\section{Structural rules and judgmental equality}

\subsection*{Capture-avoiding substitution}

When we write a substitution \(t[a/x]\), we mean the process of
\emph{replacing all free occurrences} of the variable \(x\) in the term \(t\)
by the term \(a\).  
However, care must be taken when \(t\) contains \emph{binders} (such as
\(\lambda y.\,u\) or a context declaration \(y\!:\!B\)) that introduce new variables.

\paragraph{The problem: variable capture.}

If we substitute naively, a free variable of \(a\) might become
\emph{accidentally bound} by one of these binders.
This error is called \emph{variable capture}.

\paragraph{Example (the bad case).}

Consider:
\[
t \;=\; \lambda y.\,x + y,
\qquad a \;=\; y.
\]
A naive substitution \(t[a/x]\) would yield
\[
\lambda y.\,y + y,
\]
but now the free \(y\) from \(a\) has been captured by the binder \(\lambda y.\),
changing its meaning completely.
Originally, the inner \(y\) in \(a\) referred to some outer variable,
but after substitution it refers to the bound parameter of the lambda.

\paragraph{The solution: capture-avoidance.}

Before performing substitution, we \emph{rename bound variables} in \(t\)
so that they do not clash with the free variables of \(a\).
This is called \emph{capture-avoiding substitution}.

In the example above, we first rename the bound variable \(y\) in \(t\)
to a fresh variable \(y'\):
\[
t' \;\defeq\; \lambda y'.\,x + y'.
\]
Now we can safely substitute \(a\) for \(x\):
\[
t'[a/x] \;=\; \lambda y'.\,y + y'.
\]
No variable has been captured, and the meaning is preserved.

\paragraph{Summary.}
\begin{itemize}
  \item \emph{Variable capture} occurs when a free variable in the substituting term becomes bound by a binder in the target expression.
  \item \emph{Capture-avoiding substitution} prevents this by systematically renaming bound variables before substitution.
  \item This ensures that substitution preserves the intended meaning of terms.
\end{itemize}

\subsection*{Preliminaries: explanation of \texorpdfstring{$\Delta$}{Delta} notation and substitution brackets \texorpdfstring{$[\,\cdot\,/\,\cdot\,]$}{[./.]}}

In what follows we write contexts of the form
\[
\Gamma,\, x\!:\!A,\, \Delta.
\]
Here:
\begin{itemize}
  \item \(\Gamma\) is the initial prefix of the context.
  \item \(x\!:\!A\) is the current variable declaration we are focusing on.
  \item \(\Delta\) denotes the \emph{remainder of the context} after \(x\!:\!A\).  
  It may contain additional declarations types that can depend on \(x\).  
  Formally, if
  \[
  \Delta \;\defeq\; y_1\!:\!B_1,\, y_2\!:\!B_2,\, \dots,\, y_k\!:\!B_k,
  \]
  then the complete context is
  \[
  \Gamma,\, x\!:\!A,\, y_1\!:\!B_1,\, y_2\!:\!B_2,\, \dots,\, y_k\!:\!B_k.
  \]
\end{itemize}

\paragraph{Substitution brackets.}
The notation \(t[a/x]\) means \emph{capture-avoiding substitution} of the term \(a\)
for the variable \(x\) in the expression \(t\).  
It replaces all free occurrences of \(x\) in \(t\) by \(a\), renaming bound variables when necessary to avoid name capture.

Similarly, for contexts we write \(\Delta[a/x]\) to mean “substitute \(a\) for \(x\) in every type declared in \(\Delta\)”.  
If
\[
\Delta \;\defeq\; y_1\!:\!B_1,\, y_2\!:\!B_2,\, \dots,\, y_k\!:\!B_k,
\]
then
\[
\Delta[a/x] \;\defeq\; y_1\!:\!B_1[a/x],\, y_2\!:\!B_2[a/x],\, \dots,\, y_k\!:\!B_k[a/x].
\]

\paragraph{Properties.}
\begin{itemize}
  \item If \(x\) does not occur free in \(\Delta\), then \(\Delta[a/x] = \Delta\).
  \item Substitution respects binding and avoids variable capture (by $\alpha$-conversion if necessary).
\end{itemize}
If \(\judg{\Gamma,x\!:\!A,\Delta}{b}{B}\), then \(b[a/x]\)
denotes the term obtained by substituting \(a\) for \(x\) in \(b\).

Thus, in the substitution rules, $\Delta$ represents the \emph{tail of the context}, and
the square brackets $[a/x]$ represent standard, capture-avoiding substitution applied to
terms, types, or all declarations within $\Delta$.

\subsection*{Variable, substitution, and weakening}

The following \emph{structural} principles are admissible (provable by induction on derivations) and may be used freely.

\paragraph{Variable.}
From a well-formed context, any declared variable has its declared type.
\[
\inferrule*[right=\rulename{Var}]{
  \Gamma \vdash \ctx \quad (x\!:\!A)\in\Gamma
}{
  \judg{\Gamma}{x}{A}
}
\]

\paragraph{Substitution (typing).}
If \(a:A\) is derivable in \(\Gamma\) and, under an extended context \(\Gamma,x\!:\!A,\Delta\), a judgment \(\judg{\,\cdot\,}{b}{B}\) is derivable, then we may substitute \(a\) for \(x\).
\[
\inferrule*[right=\rulename{Subst\(_1\)}]{
  \judg{\Gamma}{a}{A}
  \quad
  \judg{\Gamma,x\!:\!A,\,\Delta}{b}{B}
}{
  \judg{\Gamma,\,\Delta[a/x]}{\,b[a/x]\,}{\,B[a/x]\,}
}
\]

\paragraph{Weakening (typing).}
If \(A\) is a type in \(\Gamma\) and some judgment holds in \(\Gamma,\Delta\), we may insert a fresh, unused declaration \(x\!:\!A\) anywhere between \(\Gamma\) and \(\Delta\).
\[
\inferrule*[right=\rulename{Wkg\(_1\)}]{
  \judg{\Gamma}{A}{\Type_i}
  \quad
  \judg{\Gamma,\,\Delta}{b}{B}
}{
  \judg{\Gamma,\,x\!:\!A,\,\Delta}{b}{B}
}
\]

\paragraph{Substitution (judgmental equality).}
\[
\inferrule*[right=\rulename{Subst\(_2\)}]{
  \judg{\Gamma}{a}{A}
  \quad
  \teq{\Gamma,x\!:\!A,\,\Delta}{b}{c}{B}
}{
  \teq{\Gamma,\,\Delta[a/x]}{\,b[a/x]\,}{\,c[a/x]\,}{\,B[a/x]\,}
}
\]

\paragraph{Weakening (judgmental equality).}
\[
\inferrule*[right=\rulename{Wkg\(_2\)}]{
  \judg{\Gamma}{A}{\Type_i}
  \quad
  \teq{\Gamma,\,\Delta}{b}{c}{B}
}{
  \teq{\Gamma,\,x\!:\!A,\,\Delta}{b}{c}{B}
}
\]

\medskip
As usual, side conditions ensure \(x\notin\dom(\Gamma)\) when extending a context.

\subsection*{Judgmental equality: laws and conversion}

We assume judgmental equality \(\teq{\Gamma}{a}{b}{A}\) is an \emph{equivalence relation} and is \emph{respected by typing}. Concretely, we use the following admissible rules.

\paragraph{Equivalence laws (at a fixed type).}
\[
\resizebox{\linewidth}{!}{$
\inferrule*[right=\rulename{Refl}]{ \judg{\Gamma}{a}{A} }{ \teq{\Gamma}{a}{a}{A} }
\qquad
\inferrule*[right=\rulename{Sym}]{ \teq{\Gamma}{a}{b}{A} }{ \teq{\Gamma}{b}{a}{A} }
\qquad
\inferrule*[right=\rulename{Trans}]{ \teq{\Gamma}{a}{b}{A} \;\; \teq{\Gamma}{b}{c}{A} }{ \teq{\Gamma}{a}{c}{A} }
$}
\]

\paragraph{Conversion (type equality transports typing).}
\[
\resizebox{\linewidth}{!}{$
\inferrule*[right=\rulename{Conv-Ty}]{
  \judg{\Gamma}{a}{A}
  \quad
  \teq{\Gamma}{A}{B}{\Type_i}
}{
  \judg{\Gamma}{a}{B}
}
\qquad
\inferrule*[right=\rulename{Conv-Eq}]{
  \teq{\Gamma}{a}{b}{A}
  \quad
  \teq{\Gamma}{A}{B}{\Type_i}
}{
  \teq{\Gamma}{a}{b}{B}
}
$}
\]

\section{Type universes (re-visited)}

We postulate an infinite hierarchy of universes of types:
\[
\Type_0, \quad \Type_1, \quad \Type_2, \quad \dots
\]

Each universe is contained in the next one, and any type in \(\Type_i\)
is also a type in \(\Type_{i+1}\).
Formally, we have the following rules:

\[
\inferrule*[right=\rulename{\Type\text{-Intro}}]{
  \Gamma \vdash \ctx
}{
  \judg{\Gamma}{\Type_i}{\Type_{i+1}}
}
\qquad
\inferrule*[right=\rulename{\Type\text{-Cumul}}]{
  \judg{\Gamma}{A}{\Type_i}
}{
  \judg{\Gamma}{A}{\Type_{i+1}}
}
\]

\paragraph{Explanation.}
\begin{itemize}
  \item The first rule (\rulename{\Type-Intro}) states that each universe \(\Type_i\)
  itself has a type in the next higher universe \(\Type_{i+1}\).
  \item The second rule (\rulename{\Type-Cumul}) expresses \emph{cumulativity}:
  if a type \(A\) belongs to some universe \(\Type_i\), it is also regarded
  as a type in every higher universe.
\end{itemize}

\paragraph{Remarks.}
We set up the rules of the type theory so that whenever a typing judgment
\(\judg{\Gamma}{a}{A}\) holds, it follows that \(\judg{\Gamma}{A}{\Type_i}\)
for some universe index \(i\).  
In other words, every type \(A\) always lives in some universe \(\Type_i\).

Furthermore, judgmental equality preserves typing:  
if \(\teq{\Gamma}{a}{b}{A}\) then both \(a\) and \(b\) have type \(A\), i.e.
\[
\teq{\Gamma}{a}{b}{A} \quad\Rightarrow\quad
\judg{\Gamma}{a}{A} \;\text{and}\; \judg{\Gamma}{b}{A}.
\]

% \subsection*{Constructor congruence (example for \(\Pi\)-introduction)}

% Each term-former preserves definitional equality in its arguments. For instance, together with the \(\Pi\)-\emph{intro} rule, we use the following congruence for \(\lambda\)-abstraction:
% \[
% \resizebox{\linewidth}{!}{$
% \inferrule*[right=\rulename{\(\Pi\)-Intro-Eq}]{
%   \teq{\Gamma}{A}{A'}{\Type_i}
%   \quad
%   \teq{\Gamma,\,x\!:\!A}{B}{B'}{\Type_i}
%   \quad
%   \teq{\Gamma,\,x\!:\!A}{b}{b'}{B}
% }{
%   \teq{\Gamma}{\lambda x.\,b}{\lambda x.\,b'}{\prod_{x:A} B}
% }
% $}
% \]
% (Analogous congruence rules hold for other constructors; we omit them for brevity.)

\section{Rules associated with a type}

Each type in dependent type theory is characterized by a collection of rules that specify how it can be formed, inhabited, used, and reasoned about:

\begin{itemize}
  \item \textbf{Formation rule}, stating when the type former can be applied;
  \item \textbf{Introduction rules}, stating how to inhabit the type;
  \item \textbf{Elimination rules}, or an induction principle, stating how to use an element of the type;
  \item \textbf{Computation rules}, which are judgmental equalities explaining what happens when elimination rules are applied to results of introduction rules;
  \item (optional) \textbf{Uniqueness principles}, which are judgmental equalities explaining how every element of the type is uniquely determined by the results of elimination rules applied to it.
\end{itemize}

\section{Constructors and canonical forms}

In type theory, each type is defined by specifying its \emph{constructors}—the canonical ways of producing elements (terms) of that type.  
Constructors determine how we can \emph{build} terms of a given type, and by extension, how we can reason about or eliminate them. Elimination rules say how to take a value of that type apart (or act on it) to produce something else.

Formally, for a type \(A : \Type\), a constructor is a term-forming rule of the form
\[
\frac{\Gamma \vdash t_1 : A_1 \quad \dots \quad \Gamma \vdash t_n : A_n}{\Gamma \vdash c(t_1,\dots,t_n) : A},
\]
which specifies how a new term \(c(t_1,\dots,t_n)\) of type \(A\) can be formed from existing terms of other types \(A_1,\dots,A_n\).  
Each type comes with one or more constructors that uniquely determine its canonical inhabitants.

\paragraph{Clarifying note.}
At this stage, the expression \(c(t_1,\dots,t_n)\) should be understood
\emph{purely syntactically}: it describes how to \emph{form} a new term from
existing ones using a constructor symbol \(c\) of arity \(n\).
The parentheses indicate syntactic application of a constructor to its arguments, not functional application. \(c(t_1,\dots,t_n)\) does not denote a function being applied to arguments, but simply the construction of a new term according to the formation rule of the type. For example, for a nullary constructor (\(n=0\)) such as the unit value \(\star\),
the general rule specializes to
\[
\frac{}{\,\Gamma \vdash \star : \mathbf{1}\,},
\]
indicating that \(\star\) is a canonical term of the unit type.

Syntactically, \(c\) is a \emph{constructor symbol} for the type \(A\), not a previously defined function you prove correct. The rule \emph{declares} that supplying \(n\) arguments of the listed types yields a canonical element of \(A\).

Just like for function type formation \(A \to B\), this is a \emph{rule of the calculus}: it \emph{licenses} forming \(c(t_1,\dots,t_n)\) once the premises hold. 

\paragraph{Other examples.}

\begin{itemize}
  \item \textbf{Empty type.}  
  The empty type \(\mathbf{0}\) has \emph{no} constructors.  
  Therefore, there are no canonical terms \(a : \mathbf{0}\). This make sense, since by definition the empty type is empty.

  \item \textbf{Sum type.}  
  For two types \(A, B : \Type\), the sum type \(A + B\) has two constructors:
  \[
  \frac{a : A}{\inl(a) : A + B}
  \qquad\text{and}\qquad
  \frac{b : B}{\inr(b) : A + B}.
  \]
  The first constructor \(\inl\) injects values from the left component \(A\),
  and the second constructor \(\inr\) injects values from the right component \(B\).
  As we will see later, these two rules fully describe the canonical forms of elements of \(A + B\).
  
  \textit{Intuition.} Constructors are the primitive building blocks of each type. For instance, every term of \(A + B\) is either constructed as \(\inl(a)\) for some \(a : A\), or as \(\inr(b)\) for some \(b : B\); there are no other canonical ways to obtain a term of this type. This property allows us to reason about sum types by \emph{case analysis}, which we will define later.
\end{itemize}

\section{Non-dependent sum types and the \texorpdfstring{$\mathsf{inl}$}{inl} and \texorpdfstring{$\mathsf{inr}$}{inr} constructors}

Given two types \(A : \Type\) and \(B : \Type\), their \emph{sum type}
is written
\[
A + B : \Type.
\]
% A term of type \(A + B\) is either a term of \(A\) or a term of \(B\).
But how is $A + B$ constructed? We first need to introduce the constructors of the type $A+B$. For two types \(A, B : \Type\), the sum type \(A + B\) has two constructors:
\[
\frac{a : A}{\inl(a) : A + B}
\qquad\text{and}\qquad
\frac{b : B}{\inr(b) : A + B}.
\]
The first constructor \(\inl\) injects values into $A+B$ from the left component \(A\), and the second constructor \(\inr\) injects values into $A+B$ from the right component \(B\). These two rules fully describe the canonical forms of elements of \(A + B\).

\paragraph{Intuition.}
Constructors are the primitive building blocks of each type.  
For instance, every term of \(A + B\) is either \(\inl(a)\) for some \(a : A\), or \(\inr(b)\) for some \(b : B\); there are no other canonical ways to obtain a term of this type.

\paragraph{Reminder.} At this stage, the expression \(\inl\) should be understood \emph{purely syntactically}: it describes how to \emph{form} a new term from existing ones using a constructor symbol \( \inl \). The parentheses indicate syntactic application of a constructor to its arguments, not functional application. We have not yet introduced functions, so \( \inl \) does not denote a function being applied to arguments, but simply the construction of a new term according to the formation rule of the type.

\paragraph{Example: How \(A + B\) looks like?}

To make the idea of a sum type concrete, let us take two small finite types:
\[
A \defeq \{\mathsf{red},\, \mathsf{green}\}, 
\qquad
B \defeq \{0,\,1\}.
\]
Then the sum type \(A + B\) consists of all elements of \(A\) tagged by \(\inl\),
and all elements of \(B\) tagged by \(\inr\):
\[
A + B \;=\;
\{\inl(\mathsf{red}),\; \inl(\mathsf{green}),\;
  \inr(0),\; \inr(1)\}.
\]
Intuitively, \(\inl(a)\) means “a value coming from \(A\), left side of the sum,” and
\(\inr(b)\) means “a value coming from \(B\), right side of the sum.”. 

\bigskip
\noindent
This can be illustrated in tabular form:

\[
\begin{array}{c|c|c}
\text{Element of } A & \text{Constructor} & \text{Element of } A + B \\ \hline
\mathsf{red} & \inl & \inl(\mathsf{red}) \\
\mathsf{green} & \inl & \inl(\mathsf{green}) \\ \hline
0 & \inr & \inr(0) \\
1 & \inr & \inr(1)
\end{array}
\]
In general, every element of \(A + B\) is either of the form
\(\inl(a)\) for some \(a : A\), or \(\inr(b)\) for some \(b : B\). If desired, we can also represent the injections diagrammatically:
\[
\begin{tikzcd}
A \arrow[dr, "\inl"'] & & B \arrow[dl, "\inr"] \\
& A + B &
\end{tikzcd}
\]
This expresses that both \(A\) and \(B\) “feed into” the coproduct \(A + B\) using the constructors $\inl$ and $\inr$.

\subsection*{Rules for the non-dependent sum type \texorpdfstring{$A + B$}{A + B}}

\paragraph{Formation.}
\[
\inferrule*[right=\rulename{\(+\)-Form}]{
  \judg{\Gamma}{A}{\Type_i}
  \quad
  \judg{\Gamma}{B}{\Type_i}
}{
  \judg{\Gamma}{A + B}{\Type_i}
}
\]

\paragraph{Introduction.}
\[
\inferrule*[right=\rulename{\(+\)-Intro\(_1\)}]{
  \judg{\Gamma}{a}{A}
}{
  \judg{\Gamma}{\inl(a)}{A + B}
}
\qquad
\inferrule*[right=\rulename{\(+\)-Intro\(_2\)}]{
  \judg{\Gamma}{b}{B}
}{
  \judg{\Gamma}{\inr(b)}{A + B}
}
\]

\paragraph{Elimination (case analysis).}
\[
\inferrule*[right=\rulename{\(+\)-Elim}]{
  \judg{\Gamma}{e}{A + B}
  \quad
  \judg{\Gamma,\,x\!:\!A}{c}{C}
  \quad
  \judg{\Gamma,\,y\!:\!B}{d}{C}
}{
  \judg{\Gamma}{
    \caseof(e,\;\lambda x.\,c,\;\lambda y.\,d)
  }{C}
}
\]

\paragraph{Computation (\(\beta\)-rules).}
\[
\inferrule*[right=\rulename{\(+\)-Comp\(_1\)}]{
  \judg{\Gamma}{a}{A}
  \quad
  \judg{\Gamma,\,x\!:\!A}{c}{C}
  \quad
  \judg{\Gamma,\,y\!:\!B}{d}{C}
}{
  \teq{\Gamma}{
    \caseof(\inl(a),\;\lambda x.\,c,\;\lambda y.\,d)
  }{
    c[a/x]
  }{C}
}
\]
\[
\inferrule*[right=\rulename{\(+\)-Comp\(_2\)}]{
  \judg{\Gamma}{b}{B}
  \quad
  \judg{\Gamma,\,x\!:\!A}{c}{C}
  \quad
  \judg{\Gamma,\,y\!:\!B}{d}{C}
}{
  \teq{\Gamma}{
    \caseof(\inr(b),\;\lambda x.\,c,\;\lambda y.\,d)
  }{
    d[b/y]
  }{C}
}
\]

\paragraph{Explanation.}
\begin{itemize}
  \item \(A + B\) is the type whose terms are either of the form \(\inl(a)\) with \(a:A\),
        or \(\inr(b)\) with \(b:B\).
  \item The \emph{elimination rule} introduces the case distinction:
        to define a term of type \(C\) from \(e:A+B\),
        it suffices to specify what to do in each case.
  \item The \emph{computation rules} express that the case analysis
        reduces to the corresponding branch when the argument
        is a canonical injection.
\end{itemize}


\section{Non-dependent product types and the \texorpdfstring{$\mathsf{pair}$}{pair} constructor}

\subsection*{Definition of the product type}

Given two types \(A : \Type\) and \(B : \Type\), their \emph{product type} is written
\[
A \times B : \Type.
\]
A term of type \(A \times B\) represents a \emph{pair} consisting of one element from \(A\) and one element from \(B\).  
In other words, it corresponds to “both an \(A\) and a \(B\) together.”

\paragraph{Constructors.}
The canonical way to construct a term of the product type is by using the pairing constructor:
\[
\frac{a : A \quad b : B}{(a, b) : A \times B}.
\]
Thus, for any \(a : A\) and \(b : B\), the ordered pair \((a, b)\) is a term of \(A \times B\).

\paragraph{Eliminators (projections).}
Given a term \(p : A \times B\), we can \emph{eliminate} it by projecting its components:
\[
\frac{p : A \times B}{\pi_1(p) : A}
\qquad\text{and}\qquad
\frac{p : A \times B}{\pi_2(p) : B}.
\]
Intuitively, \(\pi_1\) extracts the first component and \(\pi_2\) extracts the second component of the pair.

\paragraph{Computation rules.}
The product type satisfies the following definitional equalities:
\[
\pi_1(a,b) \jdeq a, \qquad \pi_2(a,b) \jdeq b.
\]
That is, projecting a pair retrieves its components directly.

\paragraph{Intuition.}
If sum types \(A + B\) represent “either an \(A\) or a \(B\),”  
then product types \(A \times B\) represent “both an \(A\) and a \(B\).”  
For example, if \(A\) is the type of names and \(B\) is the type of ages,  
then \(A \times B\) is the type of \emph{name–age pairs}.

\paragraph{Example.}
Let
\[
A \defeq \{\mathsf{red},\, \mathsf{green}\}, \qquad
B \defeq \{0,\, 1\}.
\]
Then the product type \(A \times B\) consists of all possible ordered pairs:
\[
A \times B \;=\;
\{(\mathsf{red},0),\; (\mathsf{red},1),\; (\mathsf{green},0),\; (\mathsf{green},1)\}.
\]

\bigskip
\noindent
This can be visualized in a grid form:
\[
\begin{array}{c|cc}
\times & 0 & 1 \\ \hline
\mathsf{red} & (\mathsf{red},0) & (\mathsf{red},1) \\
\mathsf{green} & (\mathsf{green},0) & (\mathsf{green},1)
\end{array}
\]
Every element of \(A \times B\) is one of these four pairs.

\paragraph{Diagrammatic intuition.}
\[
\begin{tikzcd}
& A \times B \arrow[dl, "\pi_1"'] \arrow[dr, "\pi_2"] & \\
A & & B
\end{tikzcd}
\]
This expresses that the product type \(A \times B\) can be projected back to its two components via the maps \(\pi_1\) and \(\pi_2\).

\subsection*{Rules for the non-dependent product \(A \times B\)}

We record the standard \emph{formation}, \emph{introduction} (pairing),
\emph{elimination} (projections), and \emph{computation} rules for the
non-dependent product.

\paragraph{Formation.}
\[
\inferrule*[right=\rulename{\(\times\)-Form}]{
  \judg{\Gamma}{A}{\Type}
  \quad
  \judg{\Gamma}{B}{\Type}
}{
  \judg{\Gamma}{A \times B}{\Type}
}
\]

\paragraph{Introduction (pairing).}
\[
\inferrule*[right=\rulename{\(\times\)-Intro}]{
  \judg{\Gamma}{a}{A}
  \quad
  \judg{\Gamma}{b}{B}
}{
  \judg{\Gamma}{(a,b)}{A \times B}
}
\]

\paragraph{Elimination (projections).}
\[
\inferrule*[right=\rulename{\(\times\)-Elim\(_1\)}]{
  \judg{\Gamma}{p}{A \times B}
}{
  \judg{\Gamma}{\pi_1(p)}{A}
}
\qquad
\inferrule*[right=\rulename{\(\times\)-Elim\(_2\)}]{
  \judg{\Gamma}{p}{A \times B}
}{
  \judg{\Gamma}{\pi_2(p)}{B}
}
\]

\paragraph{Computation (\(\beta\)-rules for projections).}
\[
\teq{\Gamma}{\pi_1(a,b)}{a}{A}
\qquad\qquad
\teq{\Gamma}{\pi_2(a,b)}{b}{B}
\]
These specify that projecting a canonical pair returns its components.

\section{Functions and the arrow type \texorpdfstring{(\(A \to B\))}{(A -> B)}}

A \emph{function} from a type \(A\) to a type \(B\) is a term of the
\emph{function type} \(A \to B\). Intuitively, a function transforms any input \(a:A\)
into an output \(b:B\). In type theory, functions are introduced by
\emph{lambda abstraction} and used by \emph{application}.

\subsection*{Rules for the arrow type}

We present the usual four rules: \emph{formation}, \emph{introduction} (lambda),
\emph{elimination} (application), and \emph{computation} (\(\beta\)-reduction).

\paragraph{Formation.}
If \(A\) and \(B\) are types, then \(A \to B\) is a type:
\[
\inferrule*[right=\rulename{\(\to\)-Form}]{
  \judg{\Gamma}{A}{\Type}
  \quad
  \judg{\Gamma}{B}{\Type}
}{
  \judg{\Gamma}{A \to B}{\Type}
}
\]

\paragraph{Introduction (lambda abstraction).}
If under the assumption \(x:A\) we can build a term \(t:B\), then
\(\lambda x.\,t\) is a function \(A \to B\):
\[
\inferrule*[right=\rulename{\(\to\)-Intro}]{
  \judg{\Gamma,\,x\!:\!A}{t}{B}
}{
  \judg{\Gamma}{\lambda x.\,t}{A \to B}
}
\]
We often write \(\lambda x:A.\,t\) to annotate the parameter type explicitly. Note that $t$ here represents an expression that potentially depends on $x$. So, strictly, speaking $\lambda x.\,t$ is just a ``name'' of a function, it's not an explicitly stated function.

\paragraph{Elimination (application).}
Given a function \(f:A\to B\) and an argument \(a:A\), we may \emph{apply} \(f\) to \(a\):
\[
\inferrule*[right=\rulename{\(\to\)-Elim}]{
  \judg{\Gamma}{f}{A \to B}
  \quad
  \judg{\Gamma}{a}{A}
}{
  \judg{\Gamma}{f\,a}{B}
}
\]

\paragraph{Computation (\(\beta\)-reduction).}
Applying a lambda to an argument computes by capture-avoiding substitution:
\[
\inferrule*[right=\rulename{\(\to\)-Comp-\(\beta\)}]{
  \judg{\Gamma,\,x\!:\!A}{t}{B}
  \quad
  \judg{\Gamma}{a}{A}
}{
  \teq{\Gamma}{(\lambda x.\,t)\,a}{t[a/x]}{B}
}
\]
Here \(t[a/x]\) denotes capture-avoiding substitution of \(a\) for \(x\) in \(t\). This rule specifies the computation \emph{only when the function is (definitionally) a \(\lambda\)-abstraction}. In general, an application \(f\,a\) need not reduce unless \(f\) \emph{unfolds} to a lambda; otherwise \(f\,a\) is a neutral term.

\paragraph{Difference between elimination and computation.}
It is important to distinguish the \emph{elimination} rule from the \emph{computation} rule.

\begin{itemize}
  \item \textbf{Elimination rule.}  
  The elimination rule specifies \emph{how to use} or \emph{consume} a term of a given type.
  It allows us to produce something else from a value of that type.
  For function types, the elimination rule is function application:
  \[
  \inferrule*[right=\rulename{\(\to\)-Elim}]{
    \judg{\Gamma}{f}{A \to B}
    \quad
    \judg{\Gamma}{a}{A}
  }{
    \judg{\Gamma}{f\,a}{B}
  }
  \]
  This rule does not specify what $f\,a$ \emph{evaluates to}; it only states that
  such a term is well-typed.

  \item \textbf{Computation rule.}  
  The computation rule (often called the \emph{$\beta$-rule}) specifies
  \emph{what happens} when an elimination acts on an introduction.
  It defines how the expression reduces or computes.
  For function types:
  \[
  (\lambda x.\,t)\,a \;\jdeq\; t[a/x].
  \]
  That is, applying a function introduced by a $\lambda$-abstraction
  to an argument $a$ yields the function body with $a$ substituted for $x$.

\end{itemize}

Elimination rules describe \emph{how we may use} a term of a type,
while computation rules describe \emph{what happens when we use it}.
In the case of functions, elimination is application,
and computation expresses how application and $\lambda$-abstraction interact:
\[
\text{Introduction (}\lambda\text{)} \;+\; \text{Elimination (application)} \;\Rightarrow\; \text{Computation (}\beta\text{)}.
\]

\subsection*{Examples}

\paragraph{Identity and constant functions.}
\[
\mathsf{id}_A \;\defeq\; \lambda x\!:\!A.\,x \;:\; A \to A,
\qquad
\mathsf{const}_{A,B}(a) \;\defeq\; \lambda \_:\!B.\,a \;:\; B \to A.
\]
The identity can be computed as \(\mathsf{id}_A\,a \jdeq a\).

\paragraph{Conditional function.}
We define the higher-order conditional operator
\[
\mathsf{if}_A
\;\defeq\;
(\lambda b\!:\!\Bool.\,
  (\lambda t\!:\!A.\,
    (\lambda f\!:\!A.\;
      \caseof(b,\; \lambda\_.\,t,\; \lambda\_.\,f))))
\;:\;
\Bool \to (A \to (A \to A)).
\]
That is, \(\mathsf{if}_A\) is a curried function taking three arguments:
a boolean \(b:\Bool\), and two values \(t,f:A\).
Its behavior is given by
\[
\mathsf{if}_A \; b \; t \; f
\;\defeq\;
\caseof(b,\; \lambda\_.\,t,\; \lambda\_.\,f),
\]
so that
\[
\mathsf{if}_A \; \mathsf{true} \; t \; f \;\jdeq\; t,
\qquad
\mathsf{if}_A \; \mathsf{false} \;t \; f \;\jdeq\; f.
\]


\paragraph{Composition.}
Given \(f:B\to C\) and \(g:A\to B\), define
\[
f \circ g \;\defeq\; \lambda x\!:\!A.\, f\,(g\,x) \;:\; A \to C.
\]

\section{Dependent function types (\texorpdfstring{$\Pi$}{Pi}-types)}

So far we have introduced non-dependent functions \(A \to B\), where both the domain \(A\)
and codomain \(B\) are fixed types.  
In the dependent setting, however, the result type may itself depend on the input.
To capture this, we introduce the \emph{dependent function type}, written
\[
\Pi(x\!:\!A).\,B,
\]
where \(B\) may depend on the particular value \(x:A\).
Intuitively, an element of \(\Pi(x\!:\!A).\,B\) is a function that, given an input \(a:A\),
returns a term of type \(B[a/x]\).

\subsection*{Motivation}

A non-dependent function \(A \to B\) can be seen as a special case of a dependent function:
if \(B\) does not depend on \(x:A\), then
\[
\Pi(x\!:\!A).\,B \;\jdeq\; A \to B.
\]
Thus, \(\Pi\)-types generalize ordinary function types.

\paragraph{Example.}
Suppose \(A \defeq \Bool\) and define a family of types
\[
B(x) \defeq
\begin{cases}
  \Nat & \text{if } x = \mathsf{true},\\
  \String & \text{if } x = \mathsf{false}.
\end{cases}
\]
Then
\[
\Pi(x\!:\!\Bool).\,B(x)
\]
is the type of functions that return a natural number when given \(\mathsf{true}\)
and a string when given \(\mathsf{false}\).  
In this case, we can think of
\[
\Pi(x\!:\!\Bool).\,B(x) \;\jdeq\; B(\mathsf{true}) \times B(\mathsf{false})
  \;\jdeq\; \Nat \times \String,
\]
a “heterogeneous product’’ where each branch has a possibly different codomain.

\subsection*{Rules for the dependent function type}

\paragraph{Formation.}
If \(A\) is a type and \(B\) is a type that may depend on \(x:A\),
then the dependent function type \(\Pi(x\!:\!A).B\) is a type:
\[
\inferrule*[right=\rulename{\(\Pi\)-Form}]{
  \judg{\Gamma}{A}{\Type_i}
  \quad
  \judg{\Gamma,\,x\!:\!A}{B}{\Type_i}
}{
  \judg{\Gamma}{\Pi(x\!:\!A).B}{\Type_i}
}
\]

\paragraph{Introduction (dependent lambda abstraction).}
If under the assumption \(x:A\) we can derive \(b:B\),
then we may form a dependent function:
\[
\inferrule*[right=\rulename{\(\Pi\)-Intro}]{
  \judg{\Gamma,\,x\!:\!A}{b}{B}
}{
  \judg{\Gamma}{\lambda(x\!:\!A).\,b}{\Pi(x\!:\!A).B}
}
\]
Here \(\lambda(x\!:\!A).\,b\) binds all free occurrences of \(x\) in \(b\).

\paragraph{Elimination (application).}
Given a dependent function \(f : \Pi(x\!:\!A).B\) and an argument \(a:A\),
we can apply \(f\) to \(a\):
\[
\inferrule*[right=\rulename{\(\Pi\)-Elim}]{
  \judg{\Gamma}{f}{\Pi(x\!:\!A).B}
  \quad
  \judg{\Gamma}{a}{A}
}{
  \judg{\Gamma}{f(a)}{B[a/x]}
}
\]
The result of applying \(f\) to \(a\) lies in the instance of \(B\)
where every occurrence of \(x\) is replaced by \(a\).

\paragraph{Computation (\(\beta\)-rule).}
When a function introduced by \(\lambda(x\!:\!A).\,b\) is applied to an argument \(a\),
it computes by substitution:
\[
\inferrule*[right=\rulename{\(\Pi\)-Comp}]{
  \judg{\Gamma,\,x\!:\!A}{b}{B}
  \quad
  \judg{\Gamma}{a}{A}
}{
  \teq{\Gamma}{(\lambda(x\!:\!A).\,b)(a)}{b[a/x]}{B[a/x]}
}
\]
This generalizes the ordinary \(\beta\)-reduction rule of non-dependent functions.

\subsection*{Remarks}

\begin{itemize}
  \item The expression \(\lambda(x\!:\!A).\,b\) binds all free occurrences of \(x\) in \(b\).
  \item If \(B\) does not depend on \(x\), we recover the ordinary function type \(A \to B\).
  \item The computation rule ensures that applying a dependent function
        to an argument performs substitution in the codomain type.
\end{itemize}

In summary, the dependent function type \(\Pi(x\!:\!A).B\)
represents the type of functions that return a result of type \(B[a/x]\)
depending on their input \(a:A\).

\section{Dependent pair types (\texorpdfstring{$\Sigma$}{Sigma}-types)}

A dependent pair (or \emph{$\Sigma$-type}) generalizes the non-dependent product
\(A \times B\) to the setting where the second component may depend on the first.
Given a type \(A\) and a family \(B\) of types indexed by \(x:A\), we write
\[
\Sigma(x\!:\!A)\,B
\]
for the type of pairs \((a,b)\) with \(a:A\) and \(b:B[a/x]\).

\subsection*{Rules for the dependent pair type}

\paragraph{Formation.}
\[
\inferrule*[right=\rulename{$\Sigma$-Form}]{
  \judg{\Gamma}{A}{\Type_i}
  \quad
  \judg{\Gamma,\,x\!:\!A}{B}{\Type_i}
}{
  \judg{\Gamma}{\Sigma(x\!:\!A)\,B}{\Type_i}
}
\]

\paragraph{Introduction (pairing).}
\[
\inferrule*[right=\rulename{$\Sigma$-Intro}]{
  \judg{\Gamma,\,x\!:\!A}{B}{\Type_i}
  \quad
  \judg{\Gamma}{a}{A}
  \quad
  \judg{\Gamma}{b}{B[a/x]}
}{
  \judg{\Gamma}{(a,b)}{\Sigma(x\!:\!A)\,B}
}
\]

\paragraph{Elimination (dependent eliminator / induction).}
Let the \emph{motive} \(C\) be a family depending on a pair \(z:\Sigma(x\!:\!A)\,B\).
We write bindings as \(z.C\) to indicate \(z\) is bound in \(C\), and \(x,y.g\) to
indicate \(x\) and \(y\) are bound in \(g\). Then:
\[
\resizebox{\linewidth}{!}{$
\inferrule*[right=\rulename{$\Sigma$-Elim}]{
  \judg{\Gamma,\,z\!:\!\Sigma(x\!:\!A)\,B}{C}{\Type_i}
  \quad
  \judg{\Gamma,\,x\!:\!A,\,y\!:\!B}{g}{C[(x,y)/z]}
  \quad
  \judg{\Gamma}{p}{\Sigma(x\!:\!A)\,B}
}{
  \judg{\Gamma}{
    \mathsf{ind}_{\Sigma(x\!:\!A)B}\!\big(z.C,\; x,y.g,\; p\big)
  }{C[p/z]}
}
$}
\]

\paragraph{Computation (\(\beta\)-rule).}
On a canonical pair, the eliminator computes by substituting the components:
\[
\resizebox{\linewidth}{!}{$
\inferrule*[right=\rulename{$\Sigma$-Comp}]{
  \judg{\Gamma,\,z\!:\!\Sigma(x\!:\!A)\,B}{C}{\Type_i}
  \quad
  \judg{\Gamma,\,x\!:\!A,\,y\!:\!B}{g}{C[(x,y)/z]}
  \quad
  \judg{\Gamma}{a'}{A}
  \quad
  \judg{\Gamma}{b'}{B[a'/x]}
}{
  \teq{\Gamma}{
    \mathsf{ind}_{\Sigma(x\!:\!A)B}\!\big(z.C,\; x,y.g,\; (a',b')\big)
  }{
    g[a'/x,\, b'/y]
  }{C[(a',b')/z]}
}
$}
\]

\subsection*{Binding and notation}

\begin{itemize}
  \item The expression \(\Sigma(x\!:\!A)\,B\) \emph{binds} \(x\) in \(B\); free occurrences
        of \(x\) in \(B\) refer to the first component of the pair.
  \item In the eliminator, \(z.C\) binds \(z\) in \(C\), and \(x,y.g\) bind \(x\) and \(y\) in \(g\).
        We treat \(\mathsf{ind}_{\Sigma(x\!:\!A)B}\) as a primitive with these binders.
\end{itemize}

\subsection*{Non-dependent special case}

When \(B\) does not contain \(x\) free, the dependent pair collapses to the ordinary
(cartesian) product; we \emph{define}
\[
A \times B \;\defeq\; \Sigma(x\!:\!A)\,B.
\]
In this case, the introduction rule specializes to pairing \((a,b):A\times B\),
and the eliminator yields the usual projections (derivable).
% (We do not postulate a separate uniqueness principle for $\Sigma$ here.)

% \section{Dependent product types}

% Alright, now this is where it starts to get interesting. In the ordinary product \(A \times B\), a term carries \emph{both} an \(A\) and a \(B\) at once. A common situation, however, is that the second type \(B\) may \emph{depend} on the first type \(A\). This is similar to a function, but we will return to that later. For now, I will avoid using function terminology. Given a type \(A:\Type\) and a family of types \(B(x):\Type\) indexed by \(x:A\),
% the \emph{dependent product} (also called the \emph{Pi type})
% \[
% \prod_{x:A} B(x) : \Type
% \]
% collects functions that, for each input \(a:A\), return an output in the \emph{matching} type \(B(a)\).

% \paragraph{Example: explicitly populated \(A\) and \(B\).} Let us take \(A \defeq \Bool\) and define a family of types \(B(x)\) that
% depends on the boolean \(x\):
% \[
% B(x) \defeq
% \begin{cases}
%   \Nat & \text{if } x = \mathsf{true},\\
%   \String & \text{if } x = \mathsf{false}.
% \end{cases}
% \]
% We can “unroll’’ the dependent product by considering one case for each value:
% \[
% \prod_{x:\Bool} B(x)
% \;\;\jdeq\;\;
% B(\mathsf{true}) \times B(\mathsf{false})
% \;\;\jdeq\;\;
% \Nat \times \String.
% \]
% This example instance shows that
% \[
% \prod_{x:\Bool} B(x)
% \]
% acts like a “heterogeneous product’’ of the component types
% \(B(\mathsf{true})\) and \(B(\mathsf{false})\):
% each term of \(A\) (here each boolean) chooses its own type.

% \subsection*{Formation Rule}

% If \(A\) is a type, and \(B(x)\) is a type for each \(x:A\), then the dependent product is a type:
% \[
% \inferrule*[right=\rulename{\(\Pi\)-Form}]{
%   \judg{\Gamma}{A}{\Type}
%   \quad
%   \judg{\Gamma,\,x:A}{B(x)}{\Type}
% }{
%   \judg{\Gamma}{\prod_{x:A} B(x)}{\Type}
% }
% \]

% \subsection*{Connection to the arrow type}

% The arrow type is a special case of the dependent product (\emph{Pi}) type:
% if \(B\) does not depend on \(x:A\), then
% \[
% \prod_{x:A} B \;\jdeq\; A \to B.
% \]
% Thus, \(\lambda\)-abstraction and application are the non-dependent instances of
% the \(\Pi\)-introduction and \(\Pi\)-elimination rules you will see next.

\end{document}